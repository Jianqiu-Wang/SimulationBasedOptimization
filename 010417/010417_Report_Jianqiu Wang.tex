\documentclass[letterpaper]{article}
\usepackage[top=0.5in, bottom=1in, left=1in, right=1in]{geometry}
\usepackage{float}
\usepackage{graphicx}
\usepackage[procnames]{listings}
\usepackage{color}
\usepackage{caption}
\usepackage{subcaption}

%---------------------Begin of Document-----------------------------------------------------
\begin{document}
\title{ Notes of Week Jan 09, 2017}
\author{Jianqiu Wang}
\date{\today}
\maketitle
%-----------------
\section{Study of PySOT}
\subsection{Brief Introduction to PySOT}
PySOT is a tool box designed for solving both continuous and discontinuous surrogated optimization problems.
\section{Apply PySOT to Two-node Network}
Based on this paper??, we simply consider a network with one supplier and receiver. We use the case in part 7, only consider node 1 and node 4.
\subsection{Important Concepts}
\begin{itemize}
\item{On-hand inventory} The inventory in stock
\item{On-order inventory} The ordered inventory but not yet shipped
\item{Backorder} Unsatisfied order
\item{Inventory position} Amount of order we \'have\': on-hand + on-order - backorder
\end{itemize}

\subsubsection{Model Assumption}
\begin{itemize}
\item A demand-driven inventory system under base-stock policy and order rationing policy
\item General network structure for the inventory system (all the pri- mary supplier, secondary supplier and direct customer node(s) of each node are designated, if any)
instead: one supplier and one director customer node
\item Length of planning horizon: 200 days with 100 days of warm-up simulation
\item Length of review cycle for each inventory
\item Probability distributions of demands at sales regions
node 2: Normal(150,30)
\item Probability distributions of the delivery preparation times
(include but not limited to time for reprocessing, transportation,
sub-packaging, etc.) at each inventory node 
node 1: Uniform(2,4) 
\item Lower bounds for service levels at each node:
node 1: 0.70
node 2: 0.95
\item unit holding cost, unit backordering cost: 1 m.u./unit/day
\end{itemize}
\subsubsection{Objective}
\subsubsection{Source of Uncertainties}

\section{Simulation Process}
\subsection{Discrete-Event System}
When running stochastic simulation of discrete-event system, we treat them as as generalized semi-Markov processe (Stochastic Simulation, page 65). We define two sets: set $S$: states of node,
and set $E$: set of possible events that can trigger state transitions. We will determine state transitions by competing clocks: when a event $e\in E$ is scheduled, the clock runs down at predetermined rate, and when it counts to $0$, the event happens and state changes. Then usually we need to reschedule new event.
\subsection{Simulation Algorithm for a general model}
\begin{itemize}
\item{1. Initialization:} Set the simulation clock $T$ to 0. Choose the initial system state $X$ and event clock readings $\{C_i\}$.
\item{2.} Let $T=\min_iC_i$ be advanced to the time of the next event and let $I$ be the index of the clock reading that achieves this minimum.
\item{3.} Execute the logic associated with event $I$, including updating the system state $X$ and event clock $\{C_i\}$.
\item{4.} Go to Step 2.
\end{itemize}

\end{document}
